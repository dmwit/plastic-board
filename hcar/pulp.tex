\documentclass{scrreprt}
\usepackage{paralist}
\usepackage{graphicx}
\usepackage[final]{hcar}

%include polycode.fmt

\begin{document}

\begin{hcarentry}{pulp}
\report{Daniel Wagner}
\status{Not yet released}
\participants{Daniel Wagner, Michael Greenberg}
\makeheader

Anybody who has used \LaTeX\ knows that it is a fantastic tool for
typesetting; but its error reporting leaves much to be desired. Even simple
documents that use a handful of packages can produce hundreds of lines of
uninteresting output on a successful run. Reviewing this output manually to
pick out the interesting parts---like errors and warnings---is a serious
chore. Sometimes, even when you have found something of interest in the
output, one must refer to the more complete log file written to disk to
discover what really went wrong; and these can easily be thousands of lines
of uninteresting output to wade through. Moreover, as a document passes
through the writing lifecycle, different \emph{kinds} of messages are of
interest. For example, in early drafts, when text is still undergoing major
changes, fixing minor typesetting infelicities is a waste of time, though
these become important as the document enters its later stages.
Unfortunately, \LaTeX\ itself has few knobs to adjust what messages are
reported.

Pulp addresses these concerns. It includes a parser for \LaTeX\ log files
and a small but expressive configuration language for identifying which
messages are of interest. The program then shows only the interesting parts
of the log file. A typical run of pulp after successfully building a
document produces no output; this makes it very easy to spot when something
has gone wrong. Additionally, pulp provides location reporting for all
messages that gives the file name and line number that caused the error.
(Errors which do not include a line number are given a ``best guess'' by
examining nearby errors from the same file.)

Next time you want to produce a great paper, process your log with pulp!

% What's following are suggestions for the content of an entry.
%
% (WHAT IS IT?)
%
% (WHAT IS ITS STATUS? / WHAT HAS HAPPENED SINCE LAST TIME?)
%
% (CAN OTHERS GET IT?)
%
% (WHAT ARE THE IMMEDIATE PLANS?)

\FurtherReading
  \url{http://github.com/dmwit/pulp}
\end{hcarentry}

\end{document}
